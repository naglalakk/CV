\documentclass{cv_styles}
\usepackage[T1]{fontenc}
\usepackage{tabularx}
\usepackage{hyperref}
\usepackage[utf8]{inputenc}
\usepackage[icelandic]{babel}
\usepackage{titlesec}
\titleformat{\section}
  {\normalfont\Large\bfseries}{\thesection}{1em}{}[{\titlerule[0.8pt]}]
\title{Ferilskrá}
\author{Þórður Hermannsson}
\date{Netfang: doddi@kott.is | Sími: +354-690-6649}
\begin{document}
\setcounter{secnumdepth}{0} %% no numbering
\maketitle
\section{Reynsla}
\datedsubsection{Sjálfstætt starfandi}{2016 - Í dag}
Verktakavinna fyrir aðila og fyrirtæki. Verksviðið er breitt þar sem að 
ég sé sjálfur um hugmyndavinnu, forritun, uppsetningu og hýsingu á verkefnum.
Vinn að sérsmíðuðum lausnum í Python, Ruby, Node.


\datedsubsection{Tempo}{2015 - 2016}
\jobtags{Starfssvið}{kerfisstjórnun, skýjalausnir, app forritun (mobile), framendaforritun}
Til að byrja með var ég í kerfisstjórnun og sá um að uppfæra þjónustur, hélt utan um aðganga/leyfi og sá um skjölun á verkferlum. Síðar fór ég í teymi sem að sá um að byggja skýjalausnir sem að fyrirtækið átti eftir að keyra á síðar. Í lokin var ég að vinna í mobile teymi sem sá um að smíða snjallsímalausn fyrir Timesheets, tímaskráningarkerfi Tempo.

\datedsubsection{DV}{2014 - 2015}
\jobtags{Starfssvið}{kerfisstjórnun, bakendaforritun, framendaforritun}
Sá um kerfisstjórnun og forritun á dv.is, netfjölmiðil DV. Einnig hannaði ég ýmsar lausnir sem að snéru að markaðssetningu fyrirtækisins eins og áskriftartilboð.

\datedsubsection{Fancy Pants Global}{2013 - 2014}
\jobtags{Starfssvið}{bakendaforritun, framendaforritun, app forritun (mobile)}
Vinna í app lausnum fyrir iOS, snjallsíma og lófatölvur. Vann einnig í lausnum skrifaðar í C\#, ASP.NET.

\section{Verkefni}
\datedsubsection{Codecov}{2017}
\textbf{Slóð}: \href{http://www.codecov.io}{http://www.codecov.io} \newline
Codecov er fyrirtæki sem að sérhæfir sig í greiningartólum fyrir niðurstöður prófana.
Unnið er að umbætum á kerfinu og hef ég meðal annars verið að endurskrifa stóran hluta af kerfinu til þess að það keyri hraðar.
Forritunarmál: C, Python

\datedsubsection{Sumargleðin}{2015 - 2017}
\textbf{Slóð}: \href{http://www.sumargledi.is}{http://www.sumargledi.is} \newline
Sumargleðin er hátíð fyrir ungmenni í 8 - 10.bekk. Hátíðin er haldin árlega og ég hef séð um kerfið sem að fylgir þessari hátíð síðastliðin 3 ár. Þetta kerfi inniheldur meðal annars
\begin{itemize}
    \item Miðasölu með tenginu við greiðslusíðu Valitor og Netgíró
    \item Bakenda fyrir stjórnendur hátíðarinnar til þess að láta inn efni
    \item Bakenda fyrir miðahafa til þess að breyta upplýsingum
    \item Leitarvél fyrir skráða miðahafa
    \item Póstlistar, sjálfvirkar póstsendingar.
\end{itemize}
Frá því að ég tók við kerfinu höfum við unnið að því að bæta það og gera það sjálfvirkara með hverju ári. Síðast voru yfir 2000 manns sem að keyptu miða í gegnum kerfið. 
Kerfið er skrifað í Python og vefumsjónakerfinu Django.

\datedsubsection{CompetitionFeed}{2016 - 2017}
\textbf{Slóð}: \href{https://www.competitionfeed.com}{https://www.competitionfeed.com} \newline
CompetitionFeed er lausn fyrir fólk sem að vill geta fylgst með samkeppnismálum. Síðan er einskonar fréttaveita sem að sýjar út fréttir tengdar samkeppnismálum víðsvegar um heiminn. Verkefnið var gert í samstarfi við hönnuðinn Owen Hindley sem að sá alfarið um útlit og viðmótshönnun á síðunni. Kerfið inniheldur bakenda fyrir stjórnendur til þess að láta inn efni og hafa umsjón með póstlistum og notendum. Kerfið er skrifað í Ruby og vefumsjónakerfinu Rails.

\datedsubsection{SeafoodIQ}{2017}
\textbf{Slóð}: \href{http://seafoodiq.com/}{http://seafoodiq.com} \newline
Síða gerð fyrir sprotafyrirtækið SeafoodIQ. Vefumsjónakerfi: Wordpress.

\newpage
\section{Nám}
\datedsubsection{Háskólinn í Reykjavík - Tölvunarfræði}{2011 - 2014}
\datedsubsection{Listaháskóli Íslands - Tónsmíðar}{2007 - 2010}
\hfill \break
\section{Kunnátta}
\begin{tabularx}{\textwidth}{X|l}
    \textbf{Tungumál} & \textbf{Færni} \\
\hline
Python & Mjög góð \\
C & Mjög góð \\
C\# & Mjög góð \\
C++  & Góð \\
Ruby & Mjög góð \\
Javascript & Mjög góð \\
CSS & Mjög góð \\
SASS & Mjög góð \\
HTML & Mjög góð \\
Haskell & Sæmileg \\
\end{tabularx} \\
\bigbreak
\noindent
\begin{tabularx}{\textwidth}{X|l}
    \textbf{Vefumsjónarkerfi} & \textbf{Færni} \\
\hline
Django & Mjög góð \\
Rails & Mjög góð \\
Wordpress & Mjög góð
\end{tabularx}
\bigbreak
\noindent
\begin{tabularx}{\textwidth}{X|l}
    \textbf{Stýrikerfi} & \textbf{Færni} \\
\hline
Mac/OSX & Mjög góð \\
Linux & Mjög góð \\
Windows & Ágæt \\
\end{tabularx}
\end{document}
