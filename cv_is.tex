%%%%%%%%%%%%%%%%%%%%%%%%%%%%%%%%%%%%%%%%%
% Þórður Hermannsson Resume/CV
% XeLaTeX Template
% Version 1.0 (7/6/2020)
%
% This template has been downloaded from:
% http://www.LaTeXTemplates.com
%
% Original author:
% Howard Wilson (https://github.com/watsonbox/cv_template_2004) with
% extensive modifications by Vel (vel@latextemplates.com)
%
% License:
% CC BY-NC-SA 3.0 (http://creativecommons.org/licenses/by-nc-sa/3.0/)
%
%%%%%%%%%%%%%%%%%%%%%%%%%%%%%%%%%%%%%%%%%

%----------------------------------------------------------------------------------------
%	PACKAGES AND OTHER DOCUMENT CONFIGURATIONS
%----------------------------------------------------------------------------------------

\documentclass[10pt]{article} % Default font size

%%%%%%%%%%%%%%%%%%%%%%%%%%%%%%%%%%%%%%%%%
% Wilson Resume/CV
% Structure Specification File
% Version 1.0 (22/1/2015)
%
% This file has been downloaded from:
% http://www.LaTeXTemplates.com
%
% License:
% CC BY-NC-SA 3.0 (http://creativecommons.org/licenses/by-nc-sa/3.0/)
%
%%%%%%%%%%%%%%%%%%%%%%%%%%%%%%%%%%%%%%%%%

%----------------------------------------------------------------------------------------
%	PACKAGES AND OTHER DOCUMENT CONFIGURATIONS
%----------------------------------------------------------------------------------------

\usepackage[a4paper, hmargin=25mm, vmargin=30mm, top=20mm]{geometry} % Use A4 paper and set margins

\usepackage{fancyhdr} % Customize the header and footer

\usepackage{lastpage} % Required for calculating the number of pages in the document

\usepackage{hyperref} % Colors for links, text and headings

\setcounter{secnumdepth}{0} % Suppress section numbering

%\usepackage[proportional,scaled=1.064]{erewhon} % Use the Erewhon font
%\usepackage[erewhon,vvarbb,bigdelims]{newtxmath} % Use the Erewhon font
\usepackage[utf8]{inputenc} % Required for inputting international characters
\usepackage[T1]{fontenc} % Output font encoding for international characters

\usepackage{fontspec} % Required for specification of custom fonts
\setmainfont[Path = ./fonts/,
Extension = .otf,
BoldFont = Erewhon-Bold,
ItalicFont = Erewhon-Italic,
BoldItalicFont = Erewhon-BoldItalic,
SmallCapsFeatures = {Letters = SmallCaps}
]{Erewhon-Regular}

\usepackage{color} % Required for custom colors
\definecolor{slateblue}{rgb}{0.17,0.22,0.34}

\usepackage{sectsty} % Allows customization of titles
\sectionfont{\color{slateblue}} % Color section titles

\fancypagestyle{plain}{\fancyhf{}\cfoot{\thepage\ of \pageref{LastPage}}} % Define a custom page style
\pagestyle{plain} % Use the custom page style through the document
\renewcommand{\headrulewidth}{0pt} % Disable the default header rule
\renewcommand{\footrulewidth}{0pt} % Disable the default footer rule

\setlength\parindent{0pt} % Stop paragraph indentation

% Non-indenting itemize
\newenvironment{itemize-noindent}
{\setlength{\leftmargini}{0em}\begin{itemize}}
{\end{itemize}}

% Text width for tabbing environments
\newlength{\smallertextwidth}
\setlength{\smallertextwidth}{\textwidth}
\addtolength{\smallertextwidth}{-2cm}

\newcommand{\sqbullet}{~\vrule height 1ex width .8ex depth -.2ex} % Custom square bullet point definition

%----------------------------------------------------------------------------------------
%	MAIN HEADER COMMAND
%----------------------------------------------------------------------------------------

\renewcommand{\title}[1]{
{\huge{\color{slateblue}\textbf{#1}}}\\ % Header section name and color
\rule{\textwidth}{0.5mm}\\ % Rule under the header
}

%----------------------------------------------------------------------------------------
%	JOB COMMAND
%----------------------------------------------------------------------------------------

\newcommand{\job}[6]{
\begin{tabbing}
\hspace{2cm} \= \kill
\textbf{#1} \> \href{#4}{#3} \\
\textbf{#2} \>\+ \textit{#5} \\
\begin{minipage}{\smallertextwidth}
\vspace{2mm}
#6
\end{minipage}
\end{tabbing}
\vspace{2mm}
}

%----------------------------------------------------------------------------------------
%	SKILL GROUP COMMAND
%----------------------------------------------------------------------------------------

\newcommand{\skillgroup}[2]{
\begin{tabbing}
\hspace{5mm} \= \kill
\sqbullet \>\+ \textbf{#1} \\
\begin{minipage}{\smallertextwidth}
\vspace{2mm}
#2
\end{minipage}
\end{tabbing}
}

%----------------------------------------------------------------------------------------
%	INTERESTS GROUP COMMAND
%-----------------------------------------------------------------------------------------

\newcommand{\interestsgroup}[1]{
\begin{tabbing}
\hspace{5mm} \= \kill
#1
\end{tabbing}
\vspace{-10mm}
}

\newcommand{\interest}[1]{\sqbullet \> \textbf{#1}\\[3pt]} % Define a custom command for individual interests

%----------------------------------------------------------------------------------------
%	TABBED BLOCK COMMAND
%----------------------------------------------------------------------------------------

\newcommand{\tabbedblock}[1]{
\begin{tabbing}
\hspace{2cm} \= \hspace{4cm} \= \kill
#1
\end{tabbing}
}
 % Include the file specifying document layout

%----------------------------------------------------------------------------------------

\begin{document}

%----------------------------------------------------------------------------------------
%	NAME AND CONTACT INFORMATION
%----------------------------------------------------------------------------------------

\title{Þórður Hermannsson -- Ferilskrá} % Print the main header

%------------------------------------------------

\parbox{0.5\textwidth}{ % Second block
\begin{tabbing} % Enables tabbing
\hspace{3cm} \= \hspace{4cm} \= \kill % Spacing within the block
{\bf Sími} \> +(354) 6906649 \\ % Mobile phone
{\bf Netfang} \> \href{mailto:doddi@kott.is}{doddi@kott.is} \\ % Email address
{\bf Github} \> \href{https://github.com/naglalakk}{github.com/naglalakk} \\ % Github
{\bf Vefsíða} \> \href{https://donnabot.dev}{donnabot.dev} \\
\end{tabbing}}

%----------------------------------------------------------------------------------------
%	PERSONAL PROFILE
%----------------------------------------------------------------------------------------

\section{Um mig}

Ég er forritari og hef starfað sjálfstætt frá 2016. Vinnan mín hefur helst farið fram á vefnum en þar hef ég smíðað síður frá grunni og séð um viðhald og uppfærslur. Eins hef ég afskaplega gaman að því að vinna með fólki í skapandi geiranum að lausnum sem að eru krefjandi og öðruvísi.\\
Ég hef brennandi áhuga á fallaforritun og hef fjárfest mikið í lausnum sem að keyra á fallaforritunartengdum málum eins og Haskell og Purescript með góðum árangri. Ástríða mín er að skrifa góðan hugbúnað sem að virkar vel og er sannreyndur.

%----------------------------------------------------------------------------------------
%	PROJECT HISTORY SECTION
%----------------------------------------------------------------------------------------

\section{Verkefni}

\job
{Apríl 2019-}{Í dag}
{Listasafn ASÍ}
{https://www.listasafnasi.is}
{Vefforritari}
{Vefur fyrir Listasafn Alþýðusambands Íslands. Vefurinn notar Purescript fyrir framenda og Haskell fyrir bakenda. Vefurinn var unnin í nánu samstarfi við starfsfólk Listasafns ASÍ og inniheldur meðal annars sérsmíðaðann bakenda fyrir starfsfólk til þess að skrá inn upplýsingar og sérsmíðaða verslun með tengingu við greiðslugátt. \\
\rule{0mm}{5mm}Tög: Purescript, Haskell, Nix\\
\rule{0mm}{5mm}Vefsíða: https://listasafnasi.is}

\job
{Júlí 2018-}{Í dag}
{Sena/Sena Live}
{https://www.sena.is, https://www.senalive.is}
{Vefforritari}
{Sena er stærsta viðburðarfyrirtæki á Íslandi. Frá 2018 hef ég séð um umsjón á öllum vefum þeirra ásamt því að vera þeim innan handar varðandi tæknilegar lausnir. Vefirnir keyra allir á Wordpress vefumsjónarkerfinu.\\
\rule{0mm}{5mm}\textbf{Tög:} Wordpress, CMS, Apache\\
\rule{0mm}{5mm}\textbf{Vefsíða: }https://sena.is, https://senalive.is}

\job
{Júlí 2018-}{Í dag}
{Iceland Airwaves}
{https://icelandairwaves.is}
{Forritari}
{Iceland Airwaves er heimsþekkt tónlistarhátíð sem að tekur á móti fjölda gesta árlega. Vinnan mín fyrir Airwaves snýr helst að viðhaldi á vefnum þeirra sem er skrifaður í Wordpress. Eins hef ég sett upp gagnagátt (API) til þess að þjónusta þau gögn sem að tengjast hátíðinni eins og tímaáætlanir og upplýsingar um tónleikastaði. Gagnagáttin er skrifuð í Haskell.\\
\rule{0mm}{5mm}\textbf{Tög:} Wordpress, CMS, Apache, Haskell, Purescript\\
\rule{0mm}{5mm}\textbf{Vefsíða:} https://icelandairwaves.is
}

\newpage
\job
{2018}{}
{Our Atlantis}
{https://djflugveloggeimskip.com}
{Leikjaforritari}
{Our Atlantis er tölvuleikur sem að byggir á sköpunarverki Steinunnar Harðardóttur. Leikurinn snýst um að safna verum sem að verða á vegi þess sem að spilar. Hver vera er tengd við brot af sama laginu og leikmaður hefur því áhrif á tónlistina sem er í bakgrunni og framvindu lagsins sem að verurnar tilheyra. Öll tónlist og í leiknum er búin til af Steinunni og leikurinn er partur af samnefndri plötuútgáfu. Við unnum leikinn í sameiningu með Unity sem er leikjaþróunarumhverfi sem að einfaldaði mikið samvinnu á grafík og öðrum forritunarþáttum. Leikurinn hefur verið sýndur á tölvuleikjaráðstefnum á Íslandi og er aðgengilegur á vefsíðu DJ Flugvél og Geimskip. \\
\rule{0mm}{5mm}\textbf{Tög:} Unity, Game development\\
\rule{0mm}{5mm}\textbf{Vefsíða:} https://djflugveloggeimskip.com
}

\job
{Sept 2016-}{2018}
{Codecov, Holland / Ísland}
{https://www.codecov.io}
{Forritari}
{Codecov er fyrirtæki sem að sérhæfir sig í hugbúnaði sem að snýr að prófunum. Ég sá um ýmiskonar minni verkefni fyrir fyritækið sem að snéru helst að því að besta hraða á kerfinu sem að var nú þegar til staðar. Við endurskrifuðum ákveðinn part af kerfinu í C fyrir aukinn hraða. Eins sá ég um hraðamælingar og bestun á Python kóða sem að var að valda flöskuhálsi í kerfinu. \\
\rule{0mm}{5mm}\textbf{Tög:} Python, C.\\
\rule{0mm}{5mm}\textbf{Vefsíða:} https://codecov.io 
}

\job
{Jan 2014-}{Í dag}
{CompetitionFeed}
{http://www.competitionfeed.com}
{Forritari, Kerfisstjóri}
{CompetitionFeed er lausn fyrir fólk sem að hefur áhuga á því að fylgjast með fréttum tengdum samkeppnismálum. Lausnin byrjaði sem rss veita þar sem að við aðlöguðum svipað kerfi til þess að sigta út samkeppnistengdar fréttir og birta. Í seinni tíð færðist síðan svo alfarið yfir í póstlista en upprunalega kerfið sér ennþá um að safna gögnum inní viðamikinn gagnabanka af samkeppnisfréttum.\\
\rule{0mm}{5mm}\textbf{Tög:} Ruby, Javascript, Rails, Redis, Memcached, Elasticsearch, Postgresql, Haskell, Purescript\\
\rule{0mm}{5mm}\textbf{Vefsíða:} https://competitionfeed.com
}

%----------------------------------------------------------------------------------------
%	EMPLOYMENT HISTORY SECTION
%----------------------------------------------------------------------------------------

\newpage

\section{Atvinna}

\job
{Sept 2015-}{Sept 2016}
{Tempo}
{http://www.tempo.io}
{Kerfisstjórnun, Vefforritari, Mobile forritun, Bakenda forritun}
{Tempo er fyrirtæki sem að sérhæfir sig í lausnum fyrir tímastjórnun. Verkefni mín fyrir fyrirtækið voru margskonar en snéru meðal annars að kerfisstjórnun í tengslum við aðgangsupplýsingar starfsmanna, forritun á snjallsíma- appi Tempo og þróun/forritun á skýjalausn Tempo.\\
\rule{0mm}{5mm}\textbf{Tög}: Python, Javascript, Shell, iOS, Android, Docker, Postgresql\\
\rule{0mm}{5mm}\textbf{Vefsíða:} https://tempo.io
}

%------------------------------------------------

\job
{Jan 2015-}{Sept 2016}
{DV}
{http://www.dv.is}
{Kerfisstjóri, Vefforritari, Vefstjóri}
{Sá um viðhald og uppihald á dv.is sem að var skrifaður í Django vefumsjónarkerfinu.\\
\rule{0mm}{5mm}\textbf{Tög:} Python, Django, Redis, Memcached, Elasticsearch, Postgresql\\
\rule{0mm}{5mm}\textbf{Vefsíða:} https://dv.is
}

%------------------------------------------------

\job
{Maí 2014-}{Jan 2015}
{Integral Turing}
{http://www.turing.is}
{Forritari}
{Forritun á ýmsum lausnum skrifaðar í Python forritunarmálinu. Integral Turing stofnaði fyrirtækið Caritas sem var með fyrstu bílaleigunum þar sem að einstaklingar gátu leigt út bílinn sinn í skammtímaleigu. Fyrirtækið sá einnig um úthringikerfi fyrir Samfylkinguna í sveitastjórnarkosningunum 2014. \\
\rule{0mm}{5mm}\textbf{Tög:} Javascript, Python/Django, Shell, Postgresql\\
\rule{0mm}{5mm}\textbf{Vefsíða:} http://turing.is
}

%------------------------------------------------

\job
{Jan 2013-}{Maí 2014}
{Fancy Pants Global}
{https://www.linkedin.com/company/fancy-pants-global/}
{Software Developer}
{Forritun á snjallsíma- lausnum. Vinna við forritun á lausnum fyrir heilbrigðisgeirann.\\
\rule{0mm}{5mm}\textbf{Tags:} Cocoa, iOS, ASP.NET, C\#, Shell, Postgresql}

%----------------------------------------------------------------------------------------
%	EDUCATION SECTION
%----------------------------------------------------------------------------------------

\newpage
\section{Námsferill}

\tabbedblock{
\bf{2011-2014} \> Tölvunarfræði - \href{http://www.ru.is}{Háskólinn í Reykjavík} \\[5pt]
\bf{2007-2010} \> Tónsmíðar - \href{http://www.lhi.is}{Listaháskóli Íslands} \\[5pt]
}

%----------------------------------------------------------------------------------------
%	IT/COMPUTING SKILLS SECTION
%----------------------------------------------------------------------------------------

\section{Almenn tölvuþekking}

\skillgroup{Forritunarmál}
{
Haskell\\
Purescript\\
\textit{Python}\\
\textit{C}\\
C\#\\
C++\\
\textit{Ruby}\\
\textit{Javascript}
}

%------------------------------------------------

\skillgroup{Vefforritun}
{
HTML\\
\textit{HTML5\\
CSS3/SASS\\
JavaScript\\
ES6\\
CoffeeScript\\
jQuery}\\
Halogen       - Tungumál: Purescript\\
Servant       - Tungumál: Haskell\\
Ruby on Rails - Tungumál: Ruby\\
\textit{Django}        - Tungumál: Pyton\\
\textit{Apache/Nginx}}
%------------------------------------------------

\skillgroup{Annað}
{
Nix           - Build, CI/CD\\
\textit{Unity}         - Game Engine\\
\textit{Docker}        - Containerization \\
\textit{Elasticsearch} - Search engine\\
\textit{Redis}         - Cache\\
\textit{Memcached}     - Cache\\
\textit{Postgresql}    - Relational Database Management System\\
\textit{MySQL}}

\end{document}
